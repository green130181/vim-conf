\chapter{vim基本用法}

\section{配置}
\begin{code}
set nobackup "关掉备份文件
set incsearch "增量寻找
set hlsearch "设置高亮搜索
set number "显示行号
set mouse=a "使用鼠标
set encoding=utf-8
set fileencoding=utf-8
set fileencodings=ucs-bom,utf-8,gb18030,cp950,cp936 "设置文字编码自动识别
set termencoding=utf-8
set ambiwidth=double "全角字符
set showtabline=2 "显示标签栏
set laststatus=2 "状态栏在倒数第二行
set statusline=%F%m%r%h%w\ [FORMAT=%{&ff}]\ [TYPE=%Y]\ [ASCII=\%03.3b]\
[HEX=\%02.2B]\ [POS=%04l,%04v][%p%%]\ [LEN=%L]
set backspace=indent,eol,start
set vb t_vb= "当vim进行编辑时,如果命令错误,会发出一个响声,该设置去掉响声
set tabstop=4 "硬TAB
set softtabstop=4 "软TAB
set shiftwidth=4 "缩进空格数
set expandtab "空格替换TAB
set completeopt=menu,longest
set updatetime=1000
set wildignore=.svn
set cursorline "高亮行列
"set cursorcolumn
set nocompatible "不要vim模仿vi模式,建议设置,否则会有很多不兼容的问题
syntax on "开启语法高亮
filetype plugin indent on "打开文件类型检测
\end{code}

\section{vim编码}
\subsection{支持中文编码的基础}
Vim要更好地支持中文编码需要两个特性:+multi\_byte和+iconv,可以用|:version|命令检查当前使用的Vim是否支持,否则的话需要重新编译。

\subsection{影响中文编码的设置项}
Vim中有几个选项会影响对多字节编码的支持:
\begin{itemize}
    \item encoding(enc):encoding 是Vim的内部使用编码,encoding的设置会影响Vim内部的Buffer、消息文字等。在 Unix环境下,encoding的默认设置等于locale;Windows环境下会和当前代码页相同。在中文Windows环境下encoding的默认设置是cp936(GBK)
    \item fileencodings(fenc):Vim在打开文件时会根据fileencodings选项来识别文件编码,fileencodings可以同时设置多个编码,Vim会根据设置的顺序来猜测所打开文件的编码
    \item fileencoding(fencs) :Vim在保存新建文件时会根据fileencoding的设置编码来保存。如果是打开已有文件,Vim会根据打开文件时所识别的编码来保存,除非在保存时重新设置fileencoding
    \item termencodings(tenc):在终端环境下使用Vim时,通过termencoding项来告诉Vim终端所使用的编码
\end{itemize}

\subsection{Vim中的编码转换}
Vim 内部使用iconv库进行编码转换,如果这几个选项所设置的编码不一致,Vim就有可能会转换编码。打开已有文件时会从文件编码转换到 encoding所设置的编码;保存文件时会从encoding设置的编码转换到fileencoding对应的编码。经常会看到Vim提示[已转换],这是表明Vim内部作了编码转换。终端环境下使用Vim,会从termencoding设置的编码转换到encoding设置的编码。

可以用|:help encoding-values|列出Vim支持的所有编码。

\section{常用快捷键}
\begin{code}
  k          上
h   l      左  右
  j          下

^        移动到该行第一个非空格的字符处
w        向前移动一个单词,将符号或标点当作单词处理
W        向前移动一个单词,不把符号或标点当作单词处理
b        向后移动一个单词,把符号或标点当作单词处理
B        向后移动一个单词,不把符号或标点当作单词处理
0        到行首
$        到行尾
gg       到页首
G        到页末
行号+G   跳转到指定行
Ctrl+g   查询当前行信息和当前文件信息

fx       向右跳到本行字符x处(x可以是任何字符)
Fx       向左跳到本行字符x处(x可以是任何字符)

tx       和fx相同,区别是跳到字符x前
Tx       和Fx相同,区别是跳到字符x后

C-b      向上滚动一屏
C-f      向下滚动一屏
C-u      向上滚动半屏
C-d      向下滚动半屏
C-y      向上滚动一行
C-e      向下滚动一行

退出Vi

ZZ:退出vi并保存
:q! :退出vi,不保存
:wq :退出vi并保存

重复操作

.:重复上一次操作

自动补齐
\end{code}

\begin{code}
C-n      匹配下一个关键字
C-p      匹配上一个关键字

插入

o:在光标下方新开一行并将光标置于新行行首,进入插入模式。
O:同上,在光标上方。

a:在光标之后进入插入模式。
A:同上,在光标之前。


R:进入替换模式,直到按下Esc
set xxx:设置XXX选项。

在Vi中删除

x: 删除当前光标下的字符
dw:删除光标之后的单词剩余部分。
d$:删除光标之后的该行剩余部分。
dd:删除当前行。

c: 功能和d相同,区别在于完成删除操作后进入INSERT MODE
cc:也是删除当前行,然后进入INSERT MODE

更改字符
rx:将当前光标下的字符更改为x(x为任意字符) ~: 更改当前光标下的字符的大小写
键盘宏操作

qcharacter:开始录制宏,character为a到z的任意字符
q:终止录制宏
@character:调用先前录制的宏

恢复误操作

u: 撤销最后执行的命令
U: 修正之前对该行的操作
Ctrl+R:Redo

在Vi中操作Frame

\end{code}

\begin{code}
c-w c-n 增加frame
c-w c-c 减少frame
c-w c-w 切换frame
c-w c-r 交换两个frame

在Vi中查找

/ + 字符串:即可在当前文件查找相应的字符串。
继续查找同一个字符串,按n或/(ENTER),若要反向继续查找,按Shift+N。
若要逆向查找,用?代替 /
.*[]^%~$ 在Vi中具有特殊含义,若需要查找则应该加上转义字符"\"

% :查找配对的括号。
s/old/new/g:替换old为new,若没有g则表示只替换一个。
若要每个替换都向用户询问则应该用gc选项

查找的一些选项
设置高亮

:set hlsearch    设置高亮
:set nohlsearch  关闭高亮
:nohlsearch      关闭当前已经设置的高亮

增量查找

:set incsearch   设置增量查找
:set noincsearch 关闭增量查找

VIM中的块操作
\end{code}

\begin{code}
Vim支持多达26个剪贴板

:reg命令查看剪切板中的内容
  选块   先用v,C-v,V选择一块,然后用y复制,再用p粘贴。
  yy     复制当前整行
  nyy    复制当前行开始的n行内容
  ayy    将光标当前行复制进寄存器a

以上指令皆可去掉a工作,则y,p对未命名寄存器工作(所有d,c,x,y的对象都被保存在这里)。
剪切/复制/粘贴

所有删除的内容自动被保存,可以用p键粘贴

对代码自动格式化 gg=G 
\end{code}

\newpage
