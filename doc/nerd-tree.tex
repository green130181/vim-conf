\chapter{nerd-tree}

\section{功能}
NERD tree可以通过键盘或鼠标控制它以树状图显示文件系统,
它甚至允许你执行一些简单的文件系统操作
\begin{itemize}
    \item 以层次树的形式显示文件和目录
    \item 节点以不同高亮显示区分以下类型
        * 文件\\
        * 目录\\
        * 符号链接\\
        * windows .lnk文件\\
        * 只读文件\\
        * 可执行文件\\
    \item 许多(可定制的)键值映射去操纵这个树
        * 键值映射去打开/关闭/浏览目录节点\\
        * 对在新的或已存在的窗口或Tab页中打开文件的映射\\
        * 键值映射更改当前树根的路径\\
        * 键值映射导航树\\
        * \ldots\\
    \item 文可以将文件和目录添加到收藏夹
    \item 大部分NERD tree导航都可以用鼠标实现
    \item 过滤树内容(可以在运行时切换)
        * 自定义文件阻止过滤器,例如正在显示的vim备份文件\\
        * 可选显示隐藏文件(.文件)\\
        * 可选不显示文件只显示目录\\
    \item 可以自定义NERD tree窗口的大小和位置
    \item 可以调整树的结点排序方式
    \item 当你浏览文件系统的时候就会有一个文件系统的模型被创建或维护。这样做有几个优点:
        * 所有文件系统信息都被缓存了,有需要的时候只要重新读入缓存\\
        * 如果重新浏览之后访问过的tree的一部分,结点就会以上次保持的展开或合拢的样子显示\\
    \item 脚本可以记忆光标位置和窗口位置,所以你可以关闭树窗口,并且当你再次打开
        (使用NERDTreeToggle)树窗口时将会展现你关闭它时候的样子
    \item 对于多Tab,可以共享一个Tree,也可以各自拥有各自的tree,还可以混合以上两种方式
    \item 本脚本默认覆盖默认的文件浏览器(netw),所以如果你使用:edit 编辑一个目录
        NERD tree将会在当前窗口显示目录内容
    \item 一个可编程的菜单系统(模拟在一个节点上右击)
        * 一个默认的菜单插件提供了基本文件系统的操作(创建/删除/移动/复制 文件/目录)\\
    \item 提供应用程序编程接口添加你的自定义键值映射
\end{itemize}

\section{配置}
loaded\_nerd\_tree          不使用NerdTree脚本\\
NERDChristmasTree           让Tree把自己给装饰得多姿多彩漂亮点\\
NERDTreeAutoCenter          控制当光标移动超过一定距离时,是否自动将焦点调整到屏中心\\
NERDTreeAutoCenterThreshold 与NERDTreeAutoCenter配合使用\\
NERDTreeCaseSensitiveSort   排序时是否大小写敏感\\
NERDTreeChDirMode           确定是否改变Vim的CWD\\
NERDTreeHighlightCursorline 是否高亮显示光标所在行\\
NERDTreeHijackNetrw         是否使用:edit命令时打开第二NerdTree\\
NERDTreeIgnore              默认的“无视”文件\\
NERDTreeBookmarksFile       指定书签文件\\
NERDTreeMouseMode           指定鼠标模式(1.双击打开;2.单目录双文件;3.单击打开)\\
NERDTreeQuitOnOpen          打开文件后是否关闭NerdTree窗口\\
NERDTreeShowBookmarks       是否默认显示书签列表\\
NERDTreeShowFiles           是否默认显示文件\\
NERDTreeShowHidden          是否默认显示隐藏文件\\
NERDTreeShowLineNumbers     是否默认显示行号\\
NERDTreeSortOrder           排序规则\\
NERDTreeStatusline          窗口状态栏\\
NERDTreeWinPos              窗口位置('left' or 'right')\\
NERDTreeWinSize             窗口宽\\
\begin{code}
let loaded\_nerd\_tree=1
let NERDTreeHijackNetrw = 0
let NERDTreeChDirMode = 2
let NERDTreeCaseSensitiveSort = 1
let NERDTreeIgnore = [ '\^\.svn\$', '\~$', '\.o$', '\.lo$', \
'\.la$']
let NERDChristmasTree=1
let NERDTreeAutoCenter=1
let NERDTreeBookmarksFile=\$VIM.'\bundle\nerdtree\data\
\nerdbookmarks.txt'
let NERDTreeMouseMode=2
let NERDTreeShowBookmarks=1
let NERDTreeShowFiles=1
let NERDTreeShowHidden=1
let NERDTreeShowLineNumbers=1
let NERDTreeWinPos='left'
let NERDTreeWinSize=31
\end{code}

\section{快捷键}
\subsection{全局命令}
:NERDTree [start-directory | bookmark]\\
    打开一个Nerdtree,根结点由参数指定,不指定参数就是以当前目录为根结点\\
:NERDTreeFromBookmark bookmark\\
    打开一个Nerdtree,根结点由参数所指定的书签\\
:NERDTreeToggle [start-directory | bookmark]\\
    在当前Tab中如果Nerdtree已经存在,就切换显示与隐藏;\\
    如果不存在,就相当于执行:NERDTree命令\\
:NERDTreeMirror\\
    从另一个Tab中共享一个NerdTree过来(在当前Tab的Tree所作的改变也反应到原Tab中)\\
    如果总共只有一个Tree,就直接共享;如果不止一个,就会询问共享哪个\\
:NERDTreeClose\\
    在当前Tab中关闭Tree\\
我们定义了;e执行:NERDTreeToggle命令\\

\subsection{书签命令}
以下命令只在在Nerdtree的buffer中有效\\
:Bookmark name\\
    将选中结点添加到书签列表中,并命名为name(书签名不可包含空格);\\
    如与现有书签重名,则覆盖现有书签。\\
:BookmarkToRoot bookmark\\
    以指定目录书签或文件书签的父目录作为根结点显示NerdTree\\
:RevealBookmark bookmark\\
    如果指定书签已经存在于当前目录树下,打开它的上层结点并选中该书签\\
:OpenBookmark bookmark\\
    打开指定的文件。(参数必须是文件书签)\\
    如果该文件在当前的目录树下,则打开它的上层结点并选中该书签\\
:ClearBookmarks [bookmarks]\\
    清除指定书签;如未指定参数,则清除所有书签\\
:ClearAllBookmarks\\
    清除所有书签\\
:ReadBookmarks\\
    重新读入'NERDTreeBookmarksFile'中的所有书签\\

\subsection{系统快捷键}
o.......在已有窗口中打开文件、目录或书签,并跳到该窗口   \\
go......在已有窗口中打开文件、目录或书签,但不跳到该窗口 \\
t.......在新Tab中打开选中文件/书签,并跳到新Tab          \\
T.......在新Tab中打开选中文件/书签,但不跳到新Tab        \\
i.......split一个新窗口打开选中文件,并跳到该窗口        \\
gi......split一个新窗口打开选中文件,但不跳到该窗口      \\
s.......vsp一个新窗口打开选中文件,并跳到该窗口          \\
gs......vsp一个新窗口打开选中文件,但不跳到该窗口        \\
!.......执行当前文件                                     \\
O.......递归打开选中结点下的所有目录                     \\
x.......合拢选中结点的父目录                             \\
X.......递归合拢选中结点下的所有目录                     \\
e.......Edit the current dif                             \\
\\
双击......相当于NERDTree-o\\
中键......对文件相当于NERDTree-i,对目录相当于NERDTree-e\\
\\
D.......删除当前书签\\
       \\
P.......跳到根结点\\
p.......跳到父结点\\
K.......跳到当前目录下同级的第一个结点\\
J.......跳到当前目录下同级的最后一个结点\\
Ctrl-j...跳到当前目录下同级的前一个结点\\
Ctrl-k...跳到当前目录下同级的后一个结点\\
       \\
C.......将选中目录或选中文件的父目录设为根结点\\
u.......将当前根结点的父目录设为根目录,并变成合拢原根结点\\
U.......将当前根结点的父目录设为根目录,但保持展开原根结点\\
r.......递归刷新选中目录\\
R.......递归刷新根结点\\
m.......显示文件系统菜单\\
cd......将CWD设为选中目录\\
       \\
I.......切换是否显示隐藏文件\\
f.......切换是否使用文件过滤器\\
F.......切换是否显示文件\\
B.......切换是否显示书签\\
       \\
q.......关闭NerdTree窗口\\
?.......切换是否显示Quick Help\\
\\
\newpage
