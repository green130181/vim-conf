\chapter{vim插件简介}

\section{插件目录}
VIM 插件一般安装在 5 个地方, 存放插件的路径都列在“runtimepath”选项中
,我们可以使用 set 命令查看:
\begin{code}
:set runtimepath 
\end{code}
我的机器显示如下
\begin{code}
runtimepath=~/.vim,/var/lib/vim/addons,/usr/share/vim/vimfiles,\
/usr/share/vim/vim71,/usr/share/vim/vimfiles/after,\
/var/lib/vim/addons/after,~/.vim/after
\end{code}
echo \$VIMRUNTIME是/usr/share/vim/vim71\newline
echo \$VIM是/usr/share/vim\newline

多数系统上,设置的缺省值搜索五个位置:
\begin{enumerate}
	\item \$HOME/.vim你的主目录,里面有你的个人偏好
	\item \$VIM/vimfiles:系统范围的 Vim 目录,系统管理员的设置
	\item \$VIMRUNTIME: Vim 发布的文件
	\item \$VIM/vimfiles/after:系统范围 Vim 目录的 "after" 目录。为了系统管理员能够修改发布的缺省
	   值或增加设置 (很少用到)
	\item \$HOME/.vim/after:你的主目录下的 "after" 目录。为了你能够修改发布或者系统范围的设置或
	   增加个人偏好 (很少用到)
\end{enumerate}
个人理解:
\begin{enumerate}
	\item 在.vimrc里面设置set cursorline 高亮行,
	\item 在.vim/after/plugin/目录中建立my.vim设置set nocursorline后行高亮被取消
	\item 不执行第二步直接执行此步在/usr/share/vim/vim71/plugin/目录中建立my.vim设置set nocursorline后行高亮被取消	
	\item 以上结论充分说明后面的设置覆盖前面的设置
\end{enumerate}

以上提到的 5 个目录的子目录结构都是相同的。 如果你希望在其它目录里
安装插件的话,建议使用 \$VIMRUNTIME 的目录结构作为模版, 将必要的
目录结构创建完整,像这样: 
\begin{itemize}
	\item filetype.vim	根据文件名决定文件类型
	\item scripts.vim	根据文件内容决定文件类型
	\item autoload/	自动载入的脚本
	\item colors/	色彩方案文件
	\item compiler/	编辑器文件
	\item doc/		文档
	\item ftplugin/	文件类型插件
	\item indent/	缩进脚本
	\item keymap/	键盘映射表文件
	\item lang/		菜单翻译
	\item menu.vim	GUI 菜单
	\item plugin/	插件脚本 
	\item print/	打印所需的文件 
	\item spell/	拼写检查文件 
	\item syntax/	语法文件 
	\item tutor/	vimtutor 所需的文件 
\end{itemize}
注:要想查看详情可以使用:help runtimepath查看

\section{我们采用的目录结构}
所有插件直接安装到.vim目录里面,在 .vimrc进行配置和快捷键映射
\section{其他的一些事项}
\subsection{插件下载}
都是通过http://www.vim.org/中搜索下载的
还有一个网址也可以下载http://vim-scripts.org/vim/scripts.html
\subsection{安装}
直接向.vim解压即可
\subsection{帮助}
每个插件基本上都有自身的帮助文件在doc目录下,直接查看该文件即可获得详细的帮助
\subsection{外部程序}
除了vim本身外我们还使用的外部程序
\begin{itemize}
    \item ctags
    \item cscope
    \item python
\end{itemize}

\newpage
